\documentclass{article}

% Packages
\usepackage{graphicx} % For including figures
\usepackage{amsmath} % For mathematical symbols and equations
\usepackage{hyperref} % For hyperlinks
\usepackage{listings} % For including code snippets
\usepackage{float} % For precise figure placement using [H]

% Title
\title{Adversarial Robustness}
\author{Ekaterina Mozhegova}
\date{7th April}

% Begin Document
\begin{document}

\maketitle

% Abstract
The study is based on the tutorial "Adversarial Robustness - Theory and Practice".

\begin{abstract}
  % Adversarial attacks are small but malicious pertrubations over the input dataset. Often invisible to the human eye, they harshly damage the model's accuracy. These adversarial attacks present a threat to security-critical tasks, 
  % such as computer vision, natural language processing, and other. For example, in cybersecurity and healthcare domains.


  Adversarial attacks are small yet malicious perturbations applied to input datasets. Often invisible to the human eye, they can damage the internal structure of the model and 
  consequently its accuracy and reliability. These attacks pose a threat to security-critical tasks, including computer vision and natural language processing, among others.
  Therefore, the objective of the new field of study is to ensure that models remain resilient to such perturbations.  
  Zico Kolter and Aleksander Madry have observed several types of adversarial attacks and proposed methods for building robust models. They are all covered in this study. 
  
\end{abstract}

% Table of Contents
\tableofcontents
\newpage

% Include Chapters
\section{Chapter 1 - Introduction to adversarial robustness}
Let's define the model for a classification problem with the hypothesis function $h_\Theta$.  The loss function (softmax loss) of the model is the following:
$l(h_{\theta}(x_{i}), y_{i})$.




The common approach for the classification problem is to solve the optimization problem:
\[\min_{\theta} \frac{1}{m} \sum_{i=1}^{m} l(h_{\theta}(x_{i}), y_{i})\]

It is typically solved by calculating the gradient of our loss function:
\[ \Theta := \Theta - \dfrac{\alpha}{\beta} \sum_{i \in \beta} \nabla_\Theta l (h_{\theta}(x_{i}), y_{i}) \]

In standard optimization tasks, we aim to minimize the loss function. However, for creating adversarial attacks, our objective is to maximize it:
\[\max_{\hat{x}} l(h_{\Theta}(\hat{x}, y))\]

$\hat x$ here describes the adversarial example.

What we are specifically interested in is the gradient:
\[  \nabla_\Theta l (h_{\theta}(x_{i}), y_{i}) \]

Adversarial example $\hat{x}$ needs to be similar to the original input $x$ to stay meaningful, so we optimize over the pertrubation to x.
\[\max_{\delta \in \Delta} l (h_{\Theta}(x+\delta), y)\]

For intuition, $\Delta$ should be the range in which the input is the same as the original $x$.


\subsection{Targeted attacks}
Using this technique we can deceive the model into predicting an incorrect class. To achieve this, we aim to maximize the loss function for the correct label while minimizing the loss 
function for the targeted class. 

\[\text{maximize} \left( l(h_{\theta}(x + \delta), y) - l(h_{\Theta}(x + \delta), y_{\text{target}}) \right)
\]


\subsection{Adversarial risks}

The concept of adversarial risk is introduced to enhance the robustness of the model. We can evaluate adversarial risks alongside traditional empirical risks. They are estimated using a finite set of data.
\[ R_{adv}(h\theta) = E_{(x,y) \sim D} \left[ \max_{\delta \in \Delta(x)} \mathcal{L}(h\theta(x+\delta),y) \right] \]

Evaluating adversarial risks allows us to assess the accuracy of the model even in cases where it is subjected to attacks or adversarial manipulation. This evaluation can help us predict
the behaviour of our model in case of attacks. 

\[ \Theta := \Theta - \dfrac{\alpha}{|\beta|} \sum_{(x, y) \in \beta} \nabla_\Theta \max_{\delta \in \Delta(x)} l (h_{\theta}(x_{i}+\delta), y_{i}) \]


The gradient of the inner term, which involves a maximization problem, is computed as follows, taking into account Danskin's theorem.% \[\delta* = argmax_\delta \in \Delta(x) l (h_\Theta (x + \delta*), y)\]

\[\delta^* = \text{argmax}_{\delta \in \Delta(x)} \mathcal{L}(h_{\Theta}(x + \delta^*), y)\]

\[\Delta_\Theta \max_{\delta \in \Delta(x)} l (h_\Theta(x+\delta), y) = \Delta_\Theta l (h_\Theta(x + \delta^*), y)\]
\section{Chapter 2 - Linear models}
\subsection{}

\subsection{}
\section{Chapter 3 - Neural Networks}

\subsection{Aspects of Neural Networks}

The application and performance of adversarial attacks are highly relevant to neural networks.

The form of the optimization problem (inner maximization problem) remains the same:
\[\max_{\|\delta\|\leq\epsilon} \ell(h_{\theta}(x + \delta), y)\]    
What is different from the previous examples is $h(\theta)$, which now represents a neural network.

The complexity of neural networks' architecture makes them more challenging to be robust, while simultaneously rendering them susceptible to adversarial attacks.

First of all, loss surfaces do not often guide to the optimal solution, while they can be too steep, which often corresponds to local optima convergence.

Secondly, the inner maximization problem for neural networks is also more challenging due to the non-convexity of the cost surface.

Moreover, it is more difficult to navigate the cost surface.\\

To further explore the topic of adversarial attacks in neural networks, let's propose that adversarial attacks are based on two main components:

\begin{enumerate}
\item The norm of the perturbation ball.
\item The optimization method used within that norm ball.\\
\end{enumerate}

Three main approaches to adversarial attacks on neural networks exist: 
\begin{enumerate}
    \item Lower Bounding the inner optimization, 
    \item Exactly solving the inner maximization (combinatorial optimization), 
    \item Upper Bound optimization
\end{enumerate}


\subsection{Lower Bounding the inner optimization}

The Lower Bound method relying on empirical solutions is considered the most common approach.

\subsubsection{The Fast Gradient Sign Method (FGSM)}

The idea of adversarial attacks via the FGSM relies on calculating an optimal perturbation 
$\delta$ that maximizes the loss function. This calculation is based on the gradient obtained through backpropagation. 
This gradient indicates how the loss function changes when we make small adjustments to $\delta$.

Based on the gradient, we adjust $\delta$ to maximize the loss function. 
\[g = \nabla_\delta l(h_\theta(x + \delta), y)\]

This is performed by adding a fraction of 
the gradient to $\delta$, scaled by a small step size parameter $\alpha$.

\[\delta = \delta +  \alpha \cdot g\]

We adjust $\delta$ based on the sign of the gradient: when $g<0$, $\delta < 0$ and when $g>0$, $\delta > 0$. 
Thus, we have \[\delta := \epsilon \cdot \text{sign}(g)\].

However, we also need to stick to the constraints. So, we might need to project $\delta$ back into a feasible region after adjusting it.

The  $l_{\infty}$ norm ball, which selects the maximum absolute value among the elements in the set, is well-suited for this task.

We must ensure $\delta$ remains within the norm ball: $||\delta||_\infty \leq \epsilon$. 
Therefore, we project $\delta$ back into the norm ball after each adjustment.


So, then we project $\delta$ again into the norm ball.  


FGSM is specifically designed to attack under $l_\infty$ norm. 


\subsubsection{Projected Gradient Descent}


An iterative method that ensures to adjust the perturbation $\delta$ in the direction that maximizes the loss function 
while ensuring it stays within the specified constraints.

Both FGSM and PGD generate adversarial examples that are optimal since they maximize the loss function.
While FGSM is a straightforward, one-step process, which takes a single step in the direction of the goal.

Thus, we can assume PGD as an iterative FGSM, which iterates over multiple steps.

\[\delta := \text{P}(\delta + \alpha \nabla_{\delta} \ell(h_\theta(x + \delta), y))\]

$P$ is the projection of the value onto thhe ball norm.
\subsubsection{Normalized Steepest Descent}

\subsubsection{Targetted attacks}

\subsubsection{Non-$l_{\infty}$ norms}

In the methods described above, the optimization was performed under the $l_{\infty}$ norm; however, 
the other norm types can also be used for the same task.

For example, we can transfer the task to the $l_2$ norm by creating a constraint that $x+\delta$ lies in the range [0, 1].

While working with $l_2$ norm, ,we simply project normalized steepest descent for the $l_2$ ball. 

\[\delta := P_{\epsilon}\left(\delta - \alpha \frac{\nabla_{\delta} \ell(h_{\theta}(x + \delta), y)}{\|\nabla_{\delta} \ell(h_{\theta}(x + \delta), y)\|_2}\right)\]


\subsection{Exactly solving the inner maximization (combinatorial optimization)}

We assume attacks against each and every class and determine whether an adversarial example exists in the neighbourhood of some point.
s
\subsubsection{Upper and Lower bounds}
\subsection{Upper bound}


\section{Chapter 4 - Adversarial training}

We need to train a model to be robust against various types of potential attacks even in the cases when attackers know everything about te model. This will be achieved by 
minimizing the worst-case loss function. 

Let's train the model and evaluate its performance when different attacks are applied to the model. 
However, due to the computational complpexity of the exact solution method, we will test only two types of attacks:
lower bound and convex upper bounds. 

\[\min_{\theta} \frac{1}{|S|} \sum_{x,y \in S} \max_{\|\delta\| \leq \epsilon} \mathcal{L}(h_{\theta}(x + \delta), y)\]

We will analyze loss surfaces to understand the origin of robustness in models trained on a specific kind of attacks.

Loss surfaces of models trained to be robust against, for example, PGD-based adversarial attacks are smoother compared 
to those of traditionally trained models. Let's recall that a smoother curve suggests that small changes in the input data will result in relatively small 
changes in the output loss.
\end{document}
